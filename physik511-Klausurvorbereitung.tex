% Für Seitenformatierung

\documentclass[DIV=15]{scrartcl}

% Zeilenumbrüche

\parindent 0pt
\parskip 6pt

% Für deutsche Buchstaben und Synthax

\usepackage[ngerman]{babel}

% Für Auflistung mit speziellen Aufzählungszeichen

\usepackage{paralist}

% zB für \del, \dif und andere Mathebefehle

\usepackage{amsmath}
\usepackage{commath}
\usepackage{amssymb}

% Für Literatur/bibliography

\usepackage[backend=biber , style=alphabetic , hyperref=true]{biblatex}

% Für \SIunit[]{} und \num in deutschem Stil

\usepackage[output-decimal-marker={,}]{siunitx}
\DeclareSIUnit\clight{\ensuremath{c}}

% Schriftart und encoding

\usepackage[utf8]{inputenc}
% Bitstream charter als default
\usepackage[charter, greekuppercase=italicized]{mathdesign}
% Lato, als sans default
\renewcommand{\sfdefault}{fla}

% Für \sfrac{}{}, also inline-frac

\usepackage{xfrac}

% Für Einbinden von pdf-Grafiken

\usepackage{graphicx}

% Umfließen von Bildern

\usepackage{floatflt}

% Für weitere Farben

\usepackage{color}

% Für Streichen von z.B. $\rightarrow$

\usepackage{centernot}

% Für Befehl \cancel{}

\usepackage{cancel}

% Für Links nach außen und innerhalb des Dokumentes

\usepackage{hyperref}

% Für Layout von Links

\hypersetup{
	citecolor=black,
	colorlinks=true,
	linkcolor=black,
	urlcolor=blue,
}

% Verschiedene Mathematik-Hilfen

\newcommand \e[1]{\cdot10^{#1}}
\newcommand\p{\partial}

\newcommand\half{\frac 12}
\newcommand\shalf{\sfrac12}

\newcommand\skp[2]{\left\langle#1,#2\right\rangle}
\newcommand\mw[1]{\left\langle#1\right\rangle}
\newcommand \eexp[1]{\mathrm{e}^{#1}}
\newcommand \dexp[1]{\exp\del{#1}}
\newcommand \dsin[1]{\sin\del{#1}}
\newcommand \dcos[1]{\cos\del{#1}}
\newcommand \dtan[1]{\tan\del{#1}}
\newcommand \darccos[1]{\arcos\del{#1}}
\newcommand \darcsin[1]{\arcsin\del{#1}}
\newcommand \darctan[1]{\arctan\del{#1}}

% Nabla und Kombinationen von Nabla

\renewcommand\div[1]{\skp{\nabla}{#1}}
\newcommand\rot{\nabla\times}
\newcommand\grad[1]{\nabla#1}
\newcommand\laplace{\triangle}
\newcommand\dalambert{\mathop{{}\Box}\nolimits}

%Für komplexe Zahlen

\newcommand \ii{\mathrm i}
\renewcommand{\Im}{\mathop{{}\mathrm{Im}}\nolimits}
\renewcommand{\Re}{\mathop{{}\mathrm{Re}}\nolimits}

%Für Bra-Ket-Notation

\newcommand\bra[1]{\left\langle#1\right|}
\newcommand\ket[1]{\left|#1\right\rangle}
\newcommand\braket[2]{\left\langle#1\left.\vphantom{#1 #2}\right|#2\right\rangle}
\newcommand\braopket[3]{\left\langle#1\left.\vphantom{#1 #2 #3}\right|#2\left.\vphantom{#1 #2 #3}\right|#3\right\rangle}


\author{Lino Lemmer}
\title{physik511 - Klausurvorbereitung}

\begin{document}
\maketitle

\section{Beispielaufgaben}

\subsection{Relativistische Energie-Masse-Impuls-Beziehung}
\[
    E^2 = p^2c^2 + M^2c^4
\]

\subsection{Transmittierte Teilchen}

$N_0$ Teilchen werden auf einen Block mit Flächendichte $m$ geschossen. $\sigma$ sei der totale Wirkungsquerschnitt.
\begin{align*}
    N &= \sigma \cdot N_0 \cdot m
    \intertext{%
        ist die Anzahl der wechselwirkenden Teilchen, demnach werden
    }
    N_0-N &= N_0\del{1 - \sigma m}
\end{align*}
Teilchen transmittiert.

\subsection{${}^{16}_8\text{O}$}

${}^{16}_8\text{O}$ besteht aus 8 Protonen und 8 Neutronen. Ein mögliches
Isotop (gleiche Ladungszahl) ist ${}^{15}_8\text{O}$, ein Isobar (gleiche
Massenzahl) ist ${}^{16}_7\text{N}$. Zudem ist es sein eigener Spiegelkern.

Zerfiele der Kern durch $\alpha$-Zerfall, entstünde ein ${}^{12}_6\text{C}$-Kern.

\subsection{Weizsäcker'sche Massenformel}

Die Weizsäcker'sche Massenformel enthält folgende Terme. Dabei gilt die
Beziehung $A \propto R^3$.
\begin{description}
    \item[Massenterm $ZM_\text{p}+\del{A-Z}M_\text{n}+ZM_\text{e}$]
        Dieser Term enthält die reine Masse der Bestandteile.
    \item[Volumenterm $-a_VA$]
        Jedes Nukleon nimmt ein bestimmtes Volumen ein, daher ist dieser Term
        proportional zur Massenzahl $A$. Die Masse die hier fehlt, wird in
        Bindungsenergie umgewandelt.
    \item[Oberflächenterm $+a_OA^{\frac23}$]
        An der Oberfläche ist die Bindungsenergie schwächer. Da die Oberfläche
        einer Kugel proportional zum quadratischen Radius ist, ist die geht die
        Massenzahl mit $A^{\frac23}$ ein.
    \item[Coulombterm $+a_CZ\del{Z-1}A^{\frac13}$]
        Durch die Coulombabstoßung wird die Bindungsenergie ebenfalls
        reduziert. Da jedes Proton hier mit den $Z-1$ anderen Protonen
        wechselwirkt kommt eine $Z\del{Z-1}$-Abhängigkeit dazu. Da die
        Coulombkraft mit dem Abstand skaliert, ist der Term proportional zu $R$
        und damit zu $A^{\frac13}$.
    \item[Symmetrieterm $+a_S\frac{\del{A-2Z}^2}{4A}$]
        Eine starke Asymmetrie zwischen der Anzahl der Protonen und Neutronen
        wirkt destabilisierend auf den Kern. Da das Vorzeichen keine Rolle
        spielt, wird hier quadriert und zur Kompensation des Quadrats durch $A$
        geteilt.
    \item[Paarungsterm $-a_PA^{-\half}$]
        Bei Kernen mit gerader Anzahl an Protonen und Neutronen, bilden jeweils
        zwei von einer Sorte ein Paar. Dies erhöht die Bindungsenergie. Ist
        eine von beiden ungerade, verschwindet dieser Term und sind sogar beide
        ungerade, wird die Bindung geschwächt, da jeweils ein Proton und ein
        Neutron ungebunden sind.
\end{description}
Ist man nur an der Bindungsenergie interessiert, lässt man den Massenterm weg und nimmt den Rest negativ.

Die typische Bindungsenergie pro Nukleon ist einige \si{\mega\electronvolt}.

\subsection{Verhalten von Nukleonen in verschiedenen Wechselwirkungen}

Protonen und Neutronen verhalten sich bei starker Wechselwirkung gleich, bei der elektro-magnetischen Wechselwirkung hingegen unterschiedlich.
\end{document}
